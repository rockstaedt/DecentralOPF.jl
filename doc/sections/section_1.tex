\chead{\textit{Introduction}}  				
\section{Introduction}
\label{sec:intro}

The importance of renewable energy resources like photovoltaic and wind generators is undeniable in the context of reducing carbon emissions to tackle climate change. The annual capacity expansion rates of renewable energy generators underpin this. For example, the installed solar energy capacity worldwide increased from 2.7 GW in 2003 to 700 GW in 2020 \citep{ritchie2020}. Nearly the same growth rate can be found in Germany where the extension of renewable energy resources was accelerated by the Renewable Energy Law (Erneuerbaren Energie Gesetz EEG) and feed-in tariffs for photovoltaic \citep{pesch2014}. The installed photovoltaic capacities in Germany increased from 0.4 GW in 2003 to 51.5 GW in 2020 \citep{bundesnetzagentur2021}. Approximately two million photovoltaic plants are responsible for the installed capacity in 2020 \citep{bundesnetzagentur2021}.\\

According to \citet{quint2019}, renewable energy resources have a big impact on electricity transmission system planning and operating. They also state that the expansion of renewables is one of the biggest changes for transmission systems since the integration of the alternating current. In most countries the \gls{tso} is responsible for a reliable supply of electricity and for grid extensions that improve the system. The \gls{tso} takes care of balancing demand and supply by dispatching all power plants in its control area accordingly. The dispatch process is based on the costs of the power plants. Subsequently, each power plant operator has to share detailed and most often sensitive cost data with the \gls{tso}. As stated by \citet{ahlqvist2022}, this sharing of cost data is one of the main characteristics that defines a centralized electricity market. \\

A centralized market schema has a lot of disadvantages according to \citet{ahlqvist2022}. For one, producers have an incentive to overestimate their costs in order to be included in the dispatch. Second, centralized electricity markets tend to be not flexible enough to incorporate new technologies like energy storage resources or demand response. Decentralized electricity markets on the other hand have the advantage that less coordination is needed. In addition, the producer does not need to share its detailed cost information and can choose its best production plan \citep{ahlqvist2022} on its own. Thus, a need to establish decentralized processes in the context of transmission grid operating can be identified. \\

A decentralized market schema would reduce the coordination work of the \glspl{tso} and would smooth the pathway to incorporate even more renewable energy resources into the transmission grid system. Based on the current state of research, there is a lack of decentralized algorithms that can be used to solve an optimal power flow in a transmission grid system. An optimal power flow optimization is a typical, probably daily task that a \gls{tso} has to conduct to ensure the supply of electricity in its control area. Existing paper that investigate a decentralized optimal power flow often miss a detailed description of the derivation and implementation of the algorithm. Unfortunately, there is not even one paper that published the source code of the algorithm. Hence, there was a motivation to establish a decentralized algorithm that solves an optimal power flow and to publish the implementation as an open-source package that can be used and extended by others. This led to the following research questions:

\begin{enumerate}
	\item Is it possible to implement a decentralized algorithm with the help of \gls{admm} that can optimize an optimal power flow without sharing sensitive cost details?
	\item Can the algorithm be extended by energy storage resources that introduce an intertemporal component into the framework?
	\item Does the decentralized algorithm yield the same results as a centralized algorithm on a case study system?
\end{enumerate}

The thesis provides with section \ref{sec:theo} the necessary knowledge to understand the derivation and implementation of the decentralized algorithm. First, the dispatch process of the German \glspl{tso} and the concept of the optimal power flow optimization are explained. Afterwards, the fundamentals of the \gls{admm} are described and a brief summary about the current state of the research is made to confirm the mentioned gaps. Section \ref{sec:app} constitutes the main work of this thesis. In this section the fundamentals are applied to derive the mathematical formulations of a centralized optimal power flow. These equations are then further transformed into a decentralized problem formulation. Next, the implementation in Julia of both the centralized and decentralized algorithm is presented to the reader. Hereby, code listings are used to better illustrate the implementation. Finally, the centralized and decentralized algorithm are tested on a three node case system and the retrieved results evaluated.



