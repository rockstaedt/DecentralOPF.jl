\chead{\textit{Zusammenfassung}}  				
\section*{Zusammenfassung}

In den letzten Jahren haben sich die Übertragungsnetze elektrischer Energiesysteme sehr stark gewandelt. Wesentlicher Treiber dieser Veränderung war der Klimawandel und die damit verbundenen Maßnahmen zur Reduzierung der Treibhausgase. Immer mehr erneuerbare Generatoren wie zum Beispiel Solaranlagen oder Windkraftanlagen wurden in fast allen Stromnetzen weltweit integriert. Dies führt dazu, dass sich die Anzahl der Marktteilnehmer vervielfacht und die Übertragungsnetzbetreiber viele Herausforderungen lösen müssen, um weiterhin einen sicheren Netz\-betrieb zu gewährleisten. Zusätzlich werden digitale Innovationen immer mehr genutzt, weil die Kosten für leistungsfähige Technik gering sind. Eine dieser Innovationen ist der Trend zu dezentralen Systemen. Das bekanntlich beste Beispiel für dezentrale Systeme ist die Kryptowährung Bitcoin, die eine Alternative zum bisherigen Bankwesen darstellt und eine Möglichkeit bietet, Transaktionen ohne einen Intermediär durchzuführen. Die vorliegende Arbeit untersucht, inwieweit es möglich ist, einen dezentralen Algorithmus zu entwickeln, der einen optimalen Lastfluss für ein elektrisches Netzwerk berechnet. Dies ist eine typische Aufgabe, die Übertragungsnetzbetreiber durchführen müssen. In der Lastflussberechnung werden mehrere Zeitperioden und elektrische Speicher berücksichtigt. Basierend auf dem Alternating Direction Method of Multipliers Algorithmus und einer Recherche des aktuellen Stands über dezentrale Energiesystem wird ein dezentraler Algorithmus entwickelt, der einen optimalen Lastfluss berechnet, ohne dass es eine zentrale Einheit gibt, die jegliche, sensiblen Informationen der Marktteilnehmer kennt. Die Berechnung wird ausschließlich von den Marktteilnehmern ausgeführt. Der dezentrale Algorithmus wird auf eine Fallstudie mit drei Netzwerkknoten angewandt und mit den Ergebnissen einer zentralen Lastflussberechnung verglichen. Der Vergleich zeigt, dass die Ergebnis bis auf Abweichungen im Promillenbereich identisch sind. Bei der Implementierung der mathematischen Formulierungen sind einige Konvergenzprobleme aufgetreten, die durch kleine Anpassungen des Algorithmus' behoben wurden. Der Programmcode der Implementierung wurde als open-source Paket veröffentlicht. Die Herleitungen und die Implementierung des dezentralen Algorithmus' zum Lösen einer optimalen Lastflussberechnung sind in der vorliegenden Arbeit genauestens dokumentiert.