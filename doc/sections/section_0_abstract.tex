\chead{\textit{Abstract}}  				
\section*{Abstract}

Modern electricity systems have undergone a huge transformation process in the last decades. One major driver was the climate change and the involved actions to reduce carbon emissions. More and more renewable, non-dispatchable energy resources like photovoltaic and wind generators were integrated into almost every national electricity network increasing the number of market participants and opposing challenges to the transmission grid operators to sustain a reliable electricity supply. On top of these changes, the availability of high-performance technology at very low costs enables new digital innovations to be on the forerun. One of those innovations is the trend toward decentralized systems. The most famous example is certainly the cryptocurrency Bitcoin which provides an alternative to the centralized banking system and showcases a way to conduct transaction without an an intermediary. This thesis investigates whether it is possible to decentralize an optimal power flow calculation that is a very common task of every transmission grid operator. Hereby, the optimal power flow considers multi-periods and the integration of energy storage resources. Based on the Alternating Direction Method of Multipliers and a review of current papers related to decentralized electricity markets, a decentralized algorithm is developed that solves an optimal power flow without a central entity knowing all sensitive information about the market participants. All computation is done by the market participants and exchanged via an information network. The decentralized algorithm is applied to a three node case study system and the obtained results are compared to a centralized algorithm. The comparison yields that the results are nearly identical except for minor differences in the per mille range. Some convergence problems were faced while implementing the mathematical formulations. They were removed by adapting the algorithm. Finally, a decentralized algorithm to solve an optimal power flow with multi-periods including energy storage resources could be established and published as an open-source package. The derivation and implementation of this algorithm are thoroughly documented in this thesis.