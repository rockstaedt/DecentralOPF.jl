\chead{\textit{Conclusion}}  				
\section{Conclusion}

This thesis analyzed if a fully decentralized algorithm can be implemented that solves an optimal power flow optimization problem with the help of the \gls{admm}. It was shown that this is possible. Hereby, storage resources and multiple time steps could be included in the modeling framework. The decentralized algorithm was implemented in Julia. During the implementation, several convergence problems were tackled. First, it was found that the only parameter of the \gls{admm} is strongly influencing the convergence of the algorithm. Thus, it is crucial to find the correct damping parameter of the \gls{admm} that allows for a high accuracy as well as a low number of iterations. Next, the weight of the power flow penalty terms prevented the convergence. Increasing the weight from $\frac{\gamma}{2}$ to 10 solved this convergence problem. Lastly, an inaccuracy in the update process of the power flow dual variables was fixed that prevented these dual variables from converging. The decentralized algorithm was compared to a centralized algorithm on a typical three node case study system with different generators and one energy storage resource. It was shown that the decentralized algorithm yields the same results as the centralized one. The identified differences can be neglected since they are only in the per mille range.

\subsection{Contributions}

Several contributions were made while writing this thesis. An open-source package DecentralOPF.jl\footnote{~\url{www.github.com/rockstaedt/DecentralOPF.jl}} was published under MIT license. It contains the complete source code for implementing the decentralized algorithm in Julia. The source code is very well documented through section \ref{sec:app} of this thesis. Solving the optimal power flow in a decentralized manner is explained in great detail and always refers to the source code. Finally, the project structure was based on best practices in modern software development. Thus, it can be easily used and extended by others.

\subsection{Limitation and Outlook}

A few assumptions limit the impact of this thesis. The implemented modeling framework is rather simple than sophisticated, e.g., the generators do not have any ramping constraints, and there are no efficiency losses for the storages. In addition, the modeling framework does not distinguish between dispatchable and non-dispatchable resources. Thus, renewable energy resources are not fully resembled. Removing these limits would undoubtedly improve the applicability of the presented decentralized algorithm. These limits also represented initial starting points for further research. Future work could also study a method to automatically determine the damping parameter $\gamma$ based on certain conditions. This would allow the algorithm to be applied to a more extensive set of case study systems. It would also be worth looking into the implementation of regulations. It was found that while providing a decentralized algorithm, specific behavior can be induced, e.g., providing financial incentives to increase the security of the transmission network. Finally, future work should also test the decentralized algorithm on a more sophisticated case study system with more than two time steps and three nodes.